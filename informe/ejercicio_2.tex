\section{Algoritmo exacto}

\subsection{Descripción del algoritmo}
El algoritmo utilizado para obtener la $k$-PMP de un grafo $G$ consiste en construir 
todas las posibles soluciones de manera ordenada para no repetirlas mediante backtracking.
Veamos que contiene el conjunto de soluciones posibles $S$ asociado a un grafo $G = (V, E)$.
Sea $P_V$ el conjunto con todas las particiones posibles del conjunto $V$:
\begin{displaymath}
  S = \left\{p \quad | \quad p \in P_V \land \left\vert{p}\right\vert \leq k\right\}
\end{displaymath}
Se puede ver que 

\subsection{Podas y estrategias}
Podas:
\begin{itemize}
  \item no considerar particiones que utilicen menos de $k$ subconjuntos.
  \item guardar el menor obtenido hasta el momento y dejar de recorrer una rama si llego a ese valor
  \item si ya existen elementos en los $k$ subconjuntos calculo el peso de agregar cada uno de los 
    restantes al más barato. Si la suma de todos los pesos más el peso actual es mayor al mínimo
    obtenido hasta el momento dejo de recorrer esa rama.
\end{itemize}

\subsection{Complejidad temporal}


\subsection{Experimentación}
\begin{itemize}
  \item un gráfico sin podas
  \item un gráfico por poda
\end{itemize}
