\section{Ejercicio 4}

\begin{algorithm}[H]
\begin{algorithmic}[1]
\caption{HeuristicaBusquedaLocal(Grafo G, nat k)}
\STATE Vector$<$Conjunto$<$Nat$>>$ conjuntos(k, vacío)
\STATE todosLosNodosAParticion(0) \textit{//todos los nodos en la partición 0}
\STATE Bool hayMejora $\leftarrow$ true
\WHILE {hayMejora}
    \STATE hayMejora $\leftarrow$ false
    \FOR {\textbf{each} nodos}
        \STATE Bool swappeado $\leftarrow$ false
        \STATE Int subset $\leftarrow$ 0
        \WHILE {!swappeado y subset $<$ k}
            \IF{subset != particionDel(nodo) y pesoDelNodoEnParticion(actual) $>$ pesoDelNodoEnParticion(subset)}
                \STATE borroNodoDeParticion(nodo,actual)
                \STATE agregoNodoAParticion(nodo,subset)
                \STATE hayMejora $\leftarrow$ true
                \STATE swappeado $\leftarrow$ true
            \ENDIF
            \STATE subset++
        \ENDWHILE
    \ENDFOR
\ENDWHILE
\RETURN conjuntos
\end{algorithmic}
\end{algorithm}

\subsection{Análisis de complejidad temporal}

\subsection{Test de complejidad}

Construimos 100 instancias aleatorias de grafos de $n$ vértices, para cada $n = {1, ... , 100}$. Para cada instancia, se calculó el tiempo de ejecución de la heurística con parámetro $k = {10, 20, ..., 100}$ cinco veces (tomando el mínimo), para luego calcular el promedio para cada $n$ separando por $k$. Así, para cada $n$ tenemos el tiempo de ejecución promedio de cada $k$. Como esperamos que la complejidad temporal sea $O(n^2)$ se dividió la muestra por $n$, esperando ver una recta, que es lo que efectivamente ocurre. Veamos el gráfico de los resultados del test:

%\begin{figure}[H]
%    \begin{minipage}[t]{\linewidth}
%		\centering
%		\frame{\includegraphics[width=\textwidth]{ejercicio-3-%complejidad-dividida-n.jpg}}
%		\label{fig:ejercicio_3_complejidad_dividida_n}
%    \end{minipage}
%\end{figure}
%
%Podemos observar que para cada $n \leq k$, el tiempo dividido por $n$ es una constante, lo cual tiene sentido porque habíamos acotado por $O(k)$ construir la solución trivial, y $O(n) \subseteq O(k)$ porque $n \leq k$. Recién a partir de $n = k+1$ el tiempo dividido $n$ se vuelve apreciable, y como esperábamos es una recta, lo que indica que es $O(n^2)$ como era nuestra hipótesis.

