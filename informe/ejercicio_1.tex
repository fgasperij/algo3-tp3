\section{Ejercicio 1}

\subsection{Biohazard}
En primer lugar es necesario recordar brevemente de qué trataba el problema \emph{Biohazard}:
\begin{quotation}
  Una empresa de logística necesita transportar en camiones $n$ productos químicos desde una fábrica
  hacia un depósito seguro. Cualquier producto puede ser transportado en cualquiera de los camiones
  y para cada par de productos $p_i$ y $p_j$ se conoce de antemano el coeficiente de peligrosidad $h_{i,j}$
  que conlleva transportar dichos productos en el mismo camión. Los camiones tiene un umbral de peligrosidad
  que no puede ser superado. Respetando esta restricción se desea averigüar la mínima cantidad de camiones
  necesaria para transportar los productos.
\end{quotation}
Ambos problemas pertenecen a una clase de problemas conocida como optimización combinatoria. La misma
puede describirse rápidamente como la búsqueda de un objeto óptimo dentro de un conjunto finito de objetos,
donde el criterio de optimalidad está determinado por la maximización o la minimización de una función
específica. Una forma clara de describir un problema en particular de optimización combinatoria es
definiendo sus partes:
\begin{itemize}
  \item El espacio de soluciones $X$.
  \item El predicado de factibilidad $P$.
  \item El conjunto de soluciones factibles $Y$.
  \item $f$ la función objetivo que queremos maximizar o minimizar.
\end{itemize}

Comparemos las diferentes partes de los dos problemas:

\begin{center}
  \begin{tabular}{ >{\centering}m{1cm}<{\centering} | m{7cm} | m{7cm} }
      \centering Parte & \centering Biohazard & \centering $k$-PMP \tabularnewline \hline
      $X$ & 
      Tomando a los productos como elementos de un conjunto $C$ el espacio de soluciones consta de todas
      las particiones del conjunto $C$. & 
      El espacio de soluciones cuenta con todas las particiones de $V$ en un
      número de $k$ o menos conjuntos ya que puede haber particiones de nodos vacías.\tabularnewline \hline
      $P$ & La suma de las peligrosidades de cada una de las particiones no supera el umbral de peligrosidad de la instancia. & 
      En este caso el predicado factibilidad es siempre verdadero ya que cualquier partición en $k$ o menos subconjuntos
      puede ser la solución. \tabularnewline \hline
      $Y$ & Todos los elementos de $X$ que cumplan $P$. Todas las particiones que no tengan un camión con una peligrosidad mayor al umbral permitidio. & 
      En este caso $X$ es igual a $Y$ porque el predicado es siempre verdadero. \tabularnewline \hline
      $f$ & $f$($S$) = \#$S$ y se la quiere minimizar. & $f$($S$) = $\sum_{e \in E_{intraparticion}} w(e)$ y se la quiere minimizar \tabularnewline 
    \end{tabular}
\end{center}

Vemos que en ambos problemas el conjunto de soluciones factibles se corresponde con un subconjunto de las particiones de los nodos en
$k$-PMP y de los camiones en \emph{Biohazard}. Además, en los dos problemas buscamos minimizar la función objetivo.

Si modelamos el problema de los camiones con grafos una posibilidad es tomar a los productos químicos como nodos y asociar la peligrosidad
de dos productos como el peso de la arista que los une. De esta forma, al modelar una instancia del problema de los camiones obtendremos 
un grafo completo con $n = \left\vert{productos}\right\vert$ y cada arista con un peso igual a la peligrosidad entre los productos 
correspondientes a esos dos nodos. Veamos como quedaría una instancia de ejemplo:

\begin{center}
  \begin{tabular}{ c | c | c | c | c}
          & $P_1$ & $P_2$ & $P_3$ & $P_4$ \tabularnewline \hline
    $P_1$ & 0 & 1 & 2 & 3 \tabularnewline \hline
    $P_2$ & 1 & 0 & 2 & 5 \tabularnewline \hline
    $P_3$ & 2 & 1 & 0 & 3 \tabularnewline \hline
    $P_4$ & 3 & 5 & 3 & 0 \tabularnewline
  \end{tabular}
  \includegraphics[scale=0.4]{ejemplo_modelado_camiones}
\end{center}

Si resolvemos $k$-PMP en el grafo resultante obtendremos la forma de transportar los productos con tan sólo $k$ camiones
de manera tal que la peligrosidad total sea mínima. Si resolviésemos $k$-PMP en el grafo variando $k$ entre 1 y n
podríamos quedarnos con la primer $k$ que cumpla que cada una de las particiones no supera el umbral de peligrosidad. Sin embargo, no
podemos asegurar que $k$ sea efectivamente la mínima cantidad de camiones. Veamos un ejemplo para ilustrar mejor esto:

\begin{center}
  \includegraphics[scale=0.5]{camiones}
\end{center}

Supongamos que el umbral de peligrosidad es $\mu = 50$. Se ve claramente que la partición que devolvería $k$-PMP para $k = 2$ ó $k = 3$ ($k = 1$ y $k = 4$
son únicas por lo cual $k$-PMP y \emph{Biohazard} devolverían la mismo) es $S = \left\{ \left\{1, 2\right\}, \left\{3, 4\right\}\right\}$.
El peso total es $\omega (S) = 1 + 90 = 91$. Por lo tanto, el mínimo $k$ para el cual todas las particiones tienen un peso menor o igual a $\mu$ es
$k = 4$. Sin embargo, tomando $S' = \left\{ \left\{1, 4\right\}, \left\{3, 2\right\}\right\}$ tenemos que $\omega (S) < \omega (S') = 100$
aunque se cumple que $(\forall s'_i \in S' \vert \omega(s'_i) < \mu)$.

Veamos qué pasa si hacemos el camino inverso modelando $k$-PMP como \emph{Biohazard}. Utilizando la misma equivalencia que antes obtenemos
un producto por nodo y la peligrosidad entre dos productos se obtiene como el peso de la arista que une a los nodos correspondientes a esos
productos. Sin embargo, ahora debemos considerar un caso que en primera instancia quedaría afuera del modelo: cuando dos nodos no están unidos 
por ninguna arista en el grafo. Lo que haremos para que nuestro modelo también incluya a este caso será asignarles una peligrosidad igual a 0 
a dos productos cuyos nodos correspondiente no son adyacentes en el grafo. De esta forma podemos asignarles una peligrosidad pero la misma no 
cambia la instancia del problema ya que no aporta peligrosidad. Por último, resta decidir qué umbral de peligrosidad utilizar. En un primer 
momento, parece que el problema no tiene sentido plantearlo como \emph{Biohazard} porque en él buscamos la mínima cantidad de particiones y 
en $k$-PMP ese valor ya lo conocemos, es $k$. Una posibilidad es variar el umbral de peligrosidad de 0 a el peso total del grafo y notar 
los umbrales en los cuales pasa a necesitar un camión más. Podríamos pensar que si necesita $i$ camiones para que niguno supere una peligrosidad
de $\mu$ y precisa $i+1$ para que ninguno supere una peligrosidad de $\mu + 1$ entonces la partición obtenida al utilizar un umbral igual a
$\mu + 1$ se corresponde con la partición de $(i+1)$-PMP. No obstante esto es falso. Un coontraejemplo es la (TODO insertar cita a la figura) FIGURA 2
en la que con $\mu = 49$ necesitamos 3 camiones pero para $\mu = 50$ nos basta con 2 camiones pero la partición correspondiente no es 2-PMP como se
explicó antes.

\subsection{Colores, colores y colores}

Se llama coloreo válido de un grafo a una asignación $f:V \rightarrow C$ tal que:
\begin{displaymath}
f(v) \neq f(u) \quad \forall (u, v) \in E
\end{displaymath}
donde los elementos de $C$ son llamados colores. Además se denomina $k$-coloreo de un grafo G a un coloreo
válido de un grafo que usa exactamente $k$ colores, es decir $\left\vert{C}\right\vert = k$. El problema de coloreo
de un grafo consta de encontrar su coloreo mínimo, un coloreo válido que utilice la mínima cantidad de colores. Podemos
modelar el problema de coloreo de forma tal que si resolvemos $k$-PMP resolvemos coloreo de un grafo.
Sea $G = (V, E)$ el grafo al cual queremos encontrarle el coloreo mínimo. La única transformación que le haremos al grafo 
será asignarle a sus aristas peso 1:
\begin{displaymath}
\omega(v) = 1 \quad \forall v \in V
\end{displaymath}




\subsection{Un día en la vida de $k$-PMP}

\begin{itemize}
  \item pabellones de un hospital. enfermedades nodos y compatibilidad entre cada una.
\end{itemize}
