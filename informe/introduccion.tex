\addcontentsline{toc}{section}{Introducción}

\section*{Introducción}

En este trabajo nos proponemos explorar el problema denominado $k$-\textit{Partición de Mínimo Peso}, al
cual nos referiremos de ahora en más como $k$-PMP. Al ser un problema que pertenece a la clase
de NP-Completos, como veremos en la sección en la que presentamos una reducción polinomial de 
coloreo a él, no se conocen algoritmos polinomiales para resolverlo. Sin embargo, nos dedicaremos
a investigar el alcance y las limitaciones de distintas formas de abordarlo ya sea con algoritmos
exactos o heurísticas. En particular nos concentraremos en el abordaje mediante backtracking, 
como representante de los algoritmos exactos, y utilizando una metaheurística conocida como GRASP.
El problema $k$-PMP consiste en encontrar, dado un grafo simple $G = (V, E)$, un entero $k$ y una
función $\omega : E \rightarrow \mathbb{R}_+$, una $k$-partición, que tenga a lo sumo $k$ subconjuntos,
de $V$ con peso mínimo. El peso de la partición es la suma de los pesos de las aristas intrapartición,
aquellas aristas cuyos extremos se encuentran en un mismo conjunto de la partición.
\newpage
